%++++++++++++++++++++++++++++++++++++++++
% Don't modify this section unless you know what you're doing!
\documentclass[letterpaper,11pt]{article}
\usepackage{natbib}
\bibliographystyle{unsrtnat}
\usepackage{tabularx} % extra features for tabular environment
\usepackage{amsmath}  % improve math presentation
\usepackage{graphicx} % takes care of graphic including machinery
\usepackage[margin=1in,letterpaper]{geometry} % decreases margins
%\usepackage{cite} % takes care of citations
\usepackage[final]{hyperref} % adds hyper links inside the generated pdf file
\hypersetup{
	colorlinks=true,       % false: boxed links; true: colored links
	linkcolor=blue,        % color of internal links
	citecolor=blue,        % color of links to bibliography
	filecolor=magenta,     % color of file links
	urlcolor=blue         
}
%+++++++++++++++++++++++++++++++++++++++
\begin{document}

\title{Topics in Advanced Multimedia Technologies \\\textbf{On How  Video Compression Quality Affects Face And Pedestrian Detection Performance}}
\author{Ivachev Fedor 2021380027, Lin Yuhong 2021380013, Your name + ID}
\date{December 16, 2021}
\maketitle

\begin{abstract}


Typically, visual intelligent systems are trained and tested on high quality datasets, however, in practical video surveillance applications, video frames cannot be assumed to be of high quality due to video encoding, transmission and decoding associated with the limited bandwidth to which the cameras are connected, and the limitations of the cameras themselves, which usually write video in H.264. Video streaming is also affected by limited network bandwidth, and usually is performed with a low bitrate. Video compression algorithms are based on video quality indicators, which are designed to take into account the capabilities of the human visual system, that is, their task is to show the best picture for a person, with the minimum allowable video size. At the same time, when the video bitrate is lowered, object recognition works with less accuracy. In this article, we evaluate 3 modern neural network models for pedestrian and face detection at various levels of video compression. We show that existing detectors are susceptible to quality distortions arising from compression artifacts during video capture. We also provide research that avoids this disadvantage.
\end{abstract}

\section{Introduction}

Smart surveillance cameras surround us everywhere. We unlock our phone and laptop using a face scanner built into the device, our car signals about cars that are at a critical distance to it, warning us of danger, we pay for purchases in stores by looking at the sensor. In addition, on the street, in the subway, in various government buildings, there are cameras that continuously transmit the image from them to the server for processing and detecting people in the image, monitor the behavior, activities, or other changing information for the purpose of protecting people and infrastructure. These images have been used for a very long time in algorithms for recognizing people not only by their face, but also by their gait.

Nevertheless, most of these video analysis systems are subject to serious risks: with a sharp decrease in network bandwidth, less information needs to be transmitted, namely, the video bitrate is reduced. In our study, we test the hypothesis of a decrease in the accuracy of object recognition, namely, faces and pedestrians in a video with a decrease in its bitrate, and analyze the results obtained.


\section{Previous work}

One popular way to improve video quality while preserving important information is to reduce the amount of irrelevant information transmitted in the video stream by compressing more parts that do not contain semantically interesting objects. This is usually done by calculating a visual saliency map.

The \textit{Biot-Savart Law} is useful in ... [2].

A \textcolor{blue}{Helmholtz coil} is a device for ... and was named after ... These coils were widely used in ... to produce \underline{uniform magnetic fields} ...

The objective in this present lab is to ...


State, derive, and describe the important equations that you will need to use to compare theory and experiment. Include diagrams as necessary to help with visualizing variables. Leave mathematical details of your derivations on the Appendix section. Below is an example to insert a numbered equation \ref{eq1} below

\begin{equation} \label{eq1} % the label is used to reference the equation
V=\frac{8\phi\Delta\pi a^{-5}}{\sqrt{3}\lambda\alpha\cdot\delta X \cdot\Sigma}+\nabla\vec{B}+\frac{\vec{E}}{\vec{v}}+\int \psi dL
\end{equation}

where $\psi$ is the distance to the Sun in units of km, $\lambda$ is something ... Always explain each variable once introduced. Do not introduce again at a later paragraph. 

Example on how to insert an equation on a separate line, unnumbered:
$$s_f=s_0+v_0t+\frac{1}{2}at^2$$
or you can state equations or variables within the paragraph like this $v_f^2=v_i^2+2a\Delta s$ or variable $\xi$.

\section{Experiments}
Introduce all the steps and the equipment you used in the experiment. List or tabulate as necessary. Include model numbers, diagrams, schematics, all with labels. Make sure that the method is listed sufficiently such that someone else can easily replicate your experiment.

Include diagrams/photos of the experimental setup (see Figure \ref{fig1}) (Note that referencing a figure comes before actually showing the figure!), or screenshots highlighting important steps of the process. Make sure it is large enough to see the whole picture; crop the picture and remove non-important spots.

\begin{figure}[ht] 
        \centering \includegraphics[width=0.9\columnwidth]{HCoil}
        \caption{\label{fig1}Every figure MUST have a caption. Note that the figure caption is below the figure! You should also label the instruments above, say, (a) Coils, (b) DMM, etc.
        }
\end{figure}

\section{Results and Analysis}

Describe all your results after presenting them. Include tabulated data set, larger tables can also be presented on the Appendix. Here's an example to insert a table -  Table~\ref{table1} is below:

\begin{table}[ht]
\begin{center}
\caption{Every table needs a caption. Note that the table caption is on top of the table! Note the consistency of precision of table values; do not forget the errors, labels, variables, and units.}
\label{table1} 
\begin{tabular}{ccc} %change to cc for 2 columns
\hline
\multicolumn{1}{c}{Distance, $d$ (km) } & \multicolumn{1}{c}{Voltage, $V\ (\pm 0.05$ V)} & \multicolumn{1}{c}{Current, $I$\ (mA $\pm 5$\%)}\\
\hline
1.2 $\pm$ 0.2 &  0.30 & 20 \\
1.6 $\pm$ 0.4 &  0.21 & 30 \\
2.5 $\pm$ 0.1 &  0.18 & 40 \\
5.9 $\pm$ 0.2 &  0.13 & 50 \\
\hline
\end{tabular}
\end{center}
\end{table}

\textbf{Use a full page to present important plotted findings, \textcolor{red}{don't be shy!}}(See Appendix for e.g.). Your plots should have axes labels with units, error bars, legend, captions, etc.

You can also discuss the sources of errors in this section; include ways on how you may want to improve the experimental methods performed. 



\section{Conclusion}
This section should be brief, concise, but complete. Directly answer your objectives, state your findings with errors, and conclude whether or not you were successful. Briefly explain if not successful.
\cite{article1}


\bibliographystyle{plainnat}
\bibliography{references.bib}

\appendix

\section*{Appendix: Velocity measurements}

Below is an example large table; include mathematical derivations here as well.
\begin{table}[ht]
\begin{center}
\caption{Every table needs a caption.}
\label{table2} 
\begin{tabular}{cc} 
\hline
\multicolumn{1}{c}{distance (m)} & \multicolumn{1}{c}{V (km s$^-1$)} \\
\hline
0.0044151 &   0.0030871 \\
0.0021633 &   0.0021343 \\
0.0003600 &   0.0018642 \\
0.0023831 &   0.0013287 \\
0.0044151 &   0.0030871 \\
0.0021633 &   0.0021343 \\
0.0003600 &   0.0018642 \\
0.0023831 &   0.0013287 \\
0.0044151 &   0.0030871 \\
\hline
\end{tabular}
\end{center}
\end{table}

See the inserted full page plot below (Figure \ref{fig2}) for reference (sample data only, past student submission).

\begin{figure}[ht] 
        \centering \includegraphics[width=.9\columnwidth]{Capture}
        \caption{\label{fig2}Every figure MUST have a caption. Note that the figure caption is below the figure! Do not foget to include error bars on your plots.
        }
\end{figure}


\end{document}
